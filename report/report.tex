\documentclass[twocolumn]{article}

\usepackage{graphicx}
\usepackage{amsmath}
\usepackage{amsthm}
\usepackage{amssymb}
\usepackage{url}
\usepackage{multirow}
\usepackage{times}
\usepackage{fullpage}

\newcommand{\comment}[1]{}

\title{CS685: Group 14 \\
Vivaad - Identifying Controversial Articles on Wikipedia}
\author{
\begin{tabular}{ccc}
	Pankaj More & Atul Agarwal \\
	\url{pankajm@iitk.ac.in} & \url{atulag@iitk.ac.in} \\
	Dept. of CSE & Dept. of CSE\\
	\multicolumn{2}{c}{Indian Institute of Technology, Kanpur}
\end{tabular}
}
\date{Mid-sem report \\	% replace by ``initial'' or ``final'' as appropriate
\today}	% replace by actual date of submission or \today

\begin{document}

\maketitle

\begin{abstract}
	%
	Wikipedia\footnote{http://wikipedia.org} is one of the most widely used repositories of human knowledge today, with articles collaboratively edited by a diverse group of volunteer editors who are passionate and knowledgeable about specific areas. As the number of articles and editors grows at a fast pace, inevitably difference of opinion arise between editors of same articles causing the content of such articles to be controversial. Controversial articles are generally manually tagged by Wikipedia editors and span many popular and interesting topics such as religion, history and politics and a lot more. In this project, we aim to automatically identify controversial articles in Wikipedia. We will compare the accuracy with the tagged articles and further extend the approach to unmarked articles.\\
	%
\end{abstract}

\section{Introduction}
	With increasing popularity of the social technologies such as wikis, blogs, social networking sites etc., online users now can easily edit, review and publish content collaboratively. Among these social technologies, Wikipedia is arguably the most popular source of knowledge and user generated content on the Web with over 22 million articles in 285 languages and over 77 thousand active editors, among which more than 4 million are from the English Wikipedia\footnote{http://en.wikipedia.org}. It is currently ranking $6^{th}$ among top visited sites according to \url{Alexa.com}. As Wikipedia is growing at a very fast pace in terms of number of articles and editors, inevitably difference in opinion arise between editors of same articles causing the content of such articles to be controversial.\\

	Currently, Wikipedia handles controversies by allowing editors to (manually) tag entire articles as controversial, thus informing both editors and reader about the disputed reliability of such articles. There have been few recent attempts to automatically detect controversial articles \cite{Kittur:2007:HSS:1240624.1240698} \cite{conf/wsdm/VuongLSLL08} \cite{conf/ht/RadMRB12}. However, since not all articles can be manually searched for controversy, there can be still many potential untagged articles that contain controversial content. We aim to automatically identify these controversial articles for mostly two reasons. First, controversies appearing in Wikipedia articles are often a good reflection or documentation of the real world. Finding controversies in Wikipedia can therefore help the general public and scholars to understand the corresponding real world controversies better. Second, it allows moderators and editors to identify such articles quickly, thereby improving the effectiveness of the dispute resolution process by reducing the amount of effort for searching for such articles. \\

\subsection{Problem Statement}

State the problem as clearly and as formally as possible.
Explain the notations, etc.
Explain the objectives, and all the inputs.

\subsection{Related Work}

Fill in all relevant past work.

\comment{

Can also comment out paragraphs, etc.

}

\section{Algorithm or Approach}

Details of the method.

Put in a pseudo-code, etc.
Explain with figures.

\comment{

Use the following format for figures:

\begin{figure}[t]
	\centering
	\includegraphics[width=0.95\columnwidth]{figure_file}
	\caption{This figure explains this.}
	\label{fig:block}
\end{figure}

And refer as Figure \ref{fig:block}.

}

\section{Results}

Details of results, in tabular and/or graphical formats.

\comment{

\begin{table}[t]
	\centering
	\begin{tabular}{|c||cc|}
		\hline
		Header 1 & Desc 1 & Desc 2 \\
		\hline
		\hline
		Row 1 & Data 1-1 & Data 1-2 \\
		Row 2 & Data 2-1 & Data 2-2 \\
		\hline
	\end{tabular}
	\caption{Table of results.}
	\label{tab:results}
\end{table}

And refer as Table \ref{tab:results}.

}

\section{Conclusions}

Clearly state the conclusions.

Also, outline the future work.

\section*{References}

\bibliographystyle{plain}
\bibliography{refs}

\end{document}
